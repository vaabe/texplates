%%%% PANDOC DEFAULTS %%%%

% Options for packages loaded elsewhere
\PassOptionsToPackage{unicode}{hyperref}
\PassOptionsToPackage{hyphens}{url}
\PassOptionsToPackage{space}{xeCJK}
%

\documentclass[
]{article}



\usepackage{lmodern}


\usepackage{amssymb,amsmath}
\usepackage{ifxetex,ifluatex}
\ifnum 0\ifxetex 1\fi\ifluatex 1\fi=0 % if pdftex
  \usepackage[T1]{fontenc}
  \usepackage[utf8]{inputenc}
  \usepackage{textcomp} % provide euro and other symbols
\else % if luatex or xetex

  \usepackage{unicode-math}

\defaultfontfeatures{Scale=MatchLowercase}
\defaultfontfeatures[\rmfamily]{Ligatures=TeX,Scale=1}






  \ifxetex
    \usepackage{xeCJK}
    \setCJKmainfont[]{Source Han Sans CN}
  \fi


  \ifluatex
    \usepackage[]{luatexja-fontspec}
    \setmainjfont[]{Source Han Sans CN}
  \fi

\fi


% Use upquote if available, for straight quotes in verbatim environments
\IfFileExists{upquote.sty}{\usepackage{upquote}}{}
\IfFileExists{microtype.sty}{% use microtype if available
  \usepackage[]{microtype}
  \UseMicrotypeSet[protrusion]{basicmath} % disable protrusion for tt fonts
}{}


\makeatletter
\@ifundefined{KOMAClassName}{% if non-KOMA class
  \IfFileExists{parskip.sty}{%
    \usepackage{parskip}
  }{% else
    \setlength{\parindent}{0pt}
    \setlength{\parskip}{6pt plus 2pt minus 1pt}}
}{% if KOMA class
  \KOMAoptions{parskip=half}}
\makeatother

\usepackage[dvipsnames]{xcolor}

\IfFileExists{xurl.sty}{\usepackage{xurl}}{} % add URL line breaks if available
\IfFileExists{bookmark.sty}{\usepackage{bookmark}}{\usepackage{hyperref}}

\urlstyle{same} % disable monospaced font for URLs











\setlength{\emergencystretch}{3em} % prevent overfull lines
\providecommand{\tightlist}{%
  \setlength{\itemsep}{0pt}\setlength{\parskip}{0pt}}
\setcounter{secnumdepth}{-\maxdimen} % remove section numbering










%%%% MAIN CONFIG %%%%

% title
\title{\textbf{语法}}

% subtitle

% author

% date
\date{}

% colors
\definecolor{MyBlue}{rgb}{0.0, 0.0, 0.75}

\hypersetup{
	colorlinks=true,
	linkcolor=MyBlue,
	urlcolor=MyBlue
}

% urlstyle
\urlstyle{tt}

% fonts
\usepackage{geometry}
\renewcommand{\familydefault}{\sfdefault}

\DeclareRobustCommand{\[}{\begin{equation}}
\DeclareRobustCommand{\]}{\end{equation}}

\begin{document}

% page margins
\newgeometry{
	margin = .2cm, 
	top = 2.0cm, 
	bottom = 2.0cm, 
	left = 2.0cm, 
	right = 2.0cm
}

\renewcommand{\v}[1]{{\mathbf{#1}}}
\newcommand{\dv}[1]{\dot{\mathbf{#1}}}
\newcommand{\ddv}[1]{\ddot{\mathbf{#1}}}
\newcommand{\hv}[1]{\hat{\mathbf{#1}}}
\newcommand{\m}[1]{[ #1 ]}

\newcommand{\bfit}[1]{\textbf{\textit{#1}}}
\renewcommand{\t}[1]{\text{#1}}

\renewcommand{\d}{\text{d}}
\newcommand{\dd}[2]{\frac{\d #1}{\d #2}}
\newcommand{\ddd}[2]{\frac{\d^2 #1}{\d #2^2}}
\newcommand{\ddt}[1]{\frac{\d #1}{\d t}}
\newcommand{\dddt}[1]{\frac{\d^2 #1}{\d t^2}}
\newcommand{\pd}[2]{\frac{\partial #1}{\partial #2}}
\newcommand{\pdd}[2]{\frac{\partial^2 #1}{\partial #2^2}}
\newcommand{\grad}{\mathbf \nabla} 
\renewcommand{\div}{\mathbf \nabla \cdot}
\newcommand{\curl}{\mathbf \nabla \times}
\newcommand{\lap}{\nabla^2}

\newcommand{\eo}{\epsilon_0}
\newcommand{\muo}{\mu_0}
\newcommand{\Lag}{\mathcal L}
\newcommand{\Ham}{\mathcal H}
\newcommand{\degc}{^\circ \text C}
\newcommand{\avo}{6.023 \cdot 10^{23}}
\renewcommand{\P}{\text{P}}
\newcommand{\p}{\text{p}}
\newcommand{\E}{\text{E}}
\newcommand{\e}[1]{\text{e}^{#1}}

\newcommand{\bra}[1]{\langle #1 |}
\newcommand{\ket}[1]{| #1 \rangle}
\newcommand{\braket}[2]{\langle #1 | #2 \rangle}
\newcommand{\adj}{^\dagger}
\newcommand{\cj}{^*}
\newcommand{\op}[1]{\hat{#1}}

\newcommand{\bm}{\begin{bmatrix}}
\newcommand{\ebm}{\end{bmatrix}}
\newcommand{\bal}{\begin{aligned}}
\newcommand{\eal}{\end{aligned}}
\newcommand{\eq}{\begin{equation}}
\newcommand{\eeq}{\end{equation}}
\newcommand{\tl}[2]{\tag{#1} \label{#1}}

\maketitle


%%%% TABLE OF CONTENTS %%%%


\hfill

\texttt{mdpdf\ plain-语法.md\ -z\ -c}

\hypertarget{sentence-structure}{%
\subsection{sentence structure}\label{sentence-structure}}

\begin{itemize}
\item
  chinese is SVO, with some notable exceptions {[}just as in english{]}
\item
  \textbf{topicalisation:} whereby you start a sentence with the topic
  (``given'' or ``old'' information) and end with the comment (``new''
  information)

  \begin{itemize}
  \tightlist
  \item
    in this structure a DO or IDO can be moved to the front of the
    sentence
  \item
    中国菜怎么做? -\textgreater{} Zhōngguó cài zěnme zuò?
    -\textgreater{} {[}chinese food{]} how make -\textgreater{} how do
    you cook chinese food?
  \end{itemize}
\item
  \textbf{ergative structure/locative inversion:} where S is moved to O
  position and the empty S position is filled with a locative expression

  \begin{itemize}
  \tightlist
  \item
    ``XVS'' structure, where X is a locative expression
  \item
    杯子下有手机 -\textgreater{} cup-under {[}has{]} phone
    -\textgreater{} under the cup has phone {[}as opposed to ``phone is
    under the cup''{]}
  \end{itemize}
\end{itemize}

\hypertarget{copula}{%
\subsection{copula}\label{copula}}

\begin{itemize}
\item
  是 -\textgreater{} to be / is
\item
  a copula is a word that links the subject to the subject complement:

  \begin{itemize}
  \tightlist
  \item
    the cup \emph{is} under the chair
  \item
    the noun phrase ``the cup'' is the subject, the verb ``is'' is the
    copula, the locative expression ``under the chair'' is the
    predicative expression
  \end{itemize}
\item
  是 for connecting nouns

  \begin{itemize}
  \tightlist
  \item
    in chinese 是 is only used to ``connect'' nouns, i.e.~Noun 1 是 Noun
    2 -\textgreater{} Noun 1 \emph{is} Noun 2
  \item
    我是你 -\textgreater{} I am you
  \item
    是 is not used to connect a noun and an adjective, e.g.~you can't
    say {[}he{]} 是 {[}fast{]} for he \emph{is} fast. in thise case you
    need the linking word 很, hěn
  \end{itemize}
\end{itemize}

\hypertarget{ux6709-the-existential-phrase}{%
\subsection{有, the existential
phrase}\label{ux6709-the-existential-phrase}}

\begin{itemize}
\tightlist
\item
  \textbf{有 / yǒu} -\textgreater{} ``has'' / ``there
  is/exists/contains''

  \begin{itemize}
  \tightlist
  \item
    有 can be used to express existence
  \item
    to express a place ``has'' a thing, or that the thing is in the
    place
  \item
    structure: {[}X{]} {[}有{]} {[}O{]}
  \item
    我家有很多小狗 -\textgreater{} wǒ jiā yǒu hěn dūo xiǎo gǒu
    -\textgreater{} {[}my house{]} {[}has{]} very many {[}puppy{]}
    -\textgreater{} there are many puppies in my house
  \item
    日本有很多中国人 -\textgreater{} Rìběn yǒu hěn dūo Zhōngguórén
    -\textgreater{} japan {[}has{]} very many chinese people
    -\textgreater{} there are many chinese people in japan
  \item
    这里有一个问题 -\textgreater{} zhèlǐ yǒu yī gè wèntí -\textgreater{}
    here {[}has{]} a problem -\textgreater{} there is a problem here
  \end{itemize}
\item
  有 can also be used to express possession ``to have''

  \begin{itemize}
  \tightlist
  \item
    我有杯子 -\textgreater{} I have a cup
  \item
    你有房子马? -\textgreater{} you have a house?
  \end{itemize}
\end{itemize}

\hypertarget{negation}{%
\subsection{negation}\label{negation}}

\begin{itemize}
\tightlist
\item
  can negate a verb by placing 不 before it:

  \begin{itemize}
  \tightlist
  \item
    我不是你 -\textgreater{} I am not you
  \item
    她没有杯子 -\textgreater{} she doesn't have a cup
  \item
    note 有 cannot be negated with 不
  \end{itemize}
\item
  can use 没有 + {[}verb{]} to indicate an action didn't happen in the
  past

  \begin{itemize}
  \tightlist
  \item
    我没有说汉语 -\textgreater{} I didn't speak chinese
  \item
    我没有在说汉语 -\textgreater{} I wasn't speaking chinese
  \end{itemize}
\item
  using 了 to represent not anymore

  \begin{itemize}
  \tightlist
  \item
    structure: {[}bù{]} {[}verb phrase{]} {[}le{]}
  \item
    here you're using 了 to represent a change in state
  \item
    我不想吃了 -\textgreater{} I don't want to eat anymore
  \item
    你不喜欢我了马? -\textgreater{} you don't like me anymore?
  \item
    手机没电了 -\textgreater{} phone ran out of power
  \end{itemize}
\item
  \textbf{已经 / yǐjīng -\textgreater{} anymore}

  \begin{itemize}
  \tightlist
  \item
    is optional. add for emphasis.
  \item
    我已经不住这里了 -\textgreater{} I {[}anymore{]} don't live here
    -\textgreater{} I don't live here anymore
  \end{itemize}
\end{itemize}

\hypertarget{questions}{%
\subsection{questions}\label{questions}}

\begin{itemize}
\tightlist
\item
  \textbf{马 / ma} -\textgreater{} question particle

  \begin{itemize}
  \tightlist
  \item
    place at end of the sentence
  \item
    你喜欢它马? -\textgreater{} do you like it?
  \end{itemize}
\item
  \textbf{呢 / ne -\textgreater{} question particle}

  \begin{itemize}
  \tightlist
  \item
    used to ask reciprocal questions or simple questions like ``what/how
    about\ldots?''
  \item
    你好马?我很好。你呢? -\textgreater{} are you good? I'm very good.
    how about you?
  \item
    这个很好。那个呢? -\textgreater{} this one is very good. what about
    that one?
  \item
    我在家。你呢? -\textgreater{} I'm at home. what about you?
  \end{itemize}
\item
  \textbf{using 不 to ask questions by negation}

  \begin{itemize}
  \tightlist
  \item
    structure: {[}V{]} 不 {[}V{]}
  \item
    他是不是你爸爸? -\textgreater{} he is {[}not is{]} your father? (is
    he your father?)
  \item
    爸爸在不在家? -\textgreater{} father at home? {[}is the same as
    asking 爸爸在家马?{]}
  \item
    你喜不喜欢它?-\textgreater{} do you like it or not?
  \end{itemize}
\item
  \textbf{using 没有 to ask questions by negation}

  \begin{itemize}
  \tightlist
  \item
    structure: 有没有
  \item
    这里有没有人? -\textgreater{} here has {[}doesn't have{]} people?
    (is anyone here?)
  \item
    有没有人在家? -\textgreater{} have {[}doesn't have{]} people
    {[}at{]} home? (is anyone home) -\textgreater{} {[}V{]} {[}S{]}
    {[}pp{]}
  \item
    那儿有没有我那两个朋友? -\textgreater{} there has {[}doesn't
    have{]} I this two friends (are those two friends of mine there?)
  \item
    他们是不是都不在家? -\textgreater{} they are {[}not are{]} all not
    at home? (are they all not at home?)
  \end{itemize}
\item
  \textbf{asking questions about completed actions with 了马}

  \begin{itemize}
  \tightlist
  \item
    你吃饭了马? -\textgreater{} did you eat?
  \end{itemize}
\item
  \textbf{asking ``what'' with 什么}

  \begin{itemize}
  \tightlist
  \item
    structure: {[}S{]} {[}V{]} {[}什么{]} {[}noun{]}
  \item
    你喜欢吃什么菜? -\textgreater{} nǐ xǐhuan chī shénme cài
    -\textgreater{} you like to eat what food -\textgreater{} what food
    do you like to eat?
  \item
    你用什么手机? -\textgreater{} you use what phone? -\textgreater{}
    what phone do you use?
  \end{itemize}
\end{itemize}

\hypertarget{possession}{%
\subsection{possession}\label{possession}}

\begin{itemize}
\tightlist
\item
  \textbf{的 / de -\textgreater{} possessive particle}

  \begin{itemize}
  \tightlist
  \item
    我的朋友 -\textgreater{} I {[}de{]} friend -\textgreater{} my friend
  \end{itemize}
\item
  \textbf{的 can be dropped if:}

  \begin{itemize}
  \tightlist
  \item
    close relationships are involved
  \item
    institutional relationship is involved {[}school, work{]}
  \item
    我家很大 -\textgreater{} my house is very big
  \item
    这是我女朋友 -\textgreater{} this is my girlfriend
  \item
    这是我妈妈 -\textgreater{} zhè shì wǒ māma -\textgreater{} this is
    my mother
  \end{itemize}
\end{itemize}

\hypertarget{prepositional-phrases}{%
\subsection{prepositional phrases}\label{prepositional-phrases}}

\begin{itemize}
\item
  use 在 to indicate the location that a verb takes place in

  \begin{itemize}
  \tightlist
  \item
    structure: {[}S{]} {[}在{]} {[}X{]} {[}V{]} {[}O{]}
  \item
    slightly unusual structure {[}coming from english{]}
  \end{itemize}
\item
  \textbf{examples:}
\item
  她在那儿没有家人

  \begin{itemize}
  \tightlist
  \item
    tā zài nǎr méiyǒu jiārén -\textgreater{} she {[}at{]} there {[}has
    not{]} family -\textgreater{} she has no family there
  \item
    {[}S{]} {[}在{]} {[}X{]} {[}V{]} {[}O{]} (preserves SVO structure)
  \end{itemize}
\item
  我在英国没有手机

  \begin{itemize}
  \tightlist
  \item
    I {[}at England{]} don't have phone
  \item
    {[}S{]} {[}在{]} {[}X{]} {[}V{]} {[}DO{]}
  \end{itemize}
\item
  我在伦敦的朋又不叫Shinzo

  \begin{itemize}
  \tightlist
  \item
    I {[}at London{]} {[}my friend{]} not called Shinzo -\textgreater{}
    my friend in London isn't called Shinzo
  \item
    my friend in london -\textgreater{} 我在伦敦的朋又
  \end{itemize}
\end{itemize}

\hypertarget{relative-locations-in-under-above-beside-etc}{%
\subsection{relative locations {[}in, under, above, beside,
etc{]}}\label{relative-locations-in-under-above-beside-etc}}

\begin{itemize}
\tightlist
\item
  \textbf{在 with ``big'' locations}

  \begin{itemize}
  \tightlist
  \item
    structure: 在 {[}X{]}
  \item
    在中国 -\textgreater{} in China
  \item
    在纽约 -\textgreater{} in New York
  \end{itemize}
\item
  \textbf{在 + ``preposition'' for relative locations}

  \begin{itemize}
  \tightlist
  \item
    structure: 在 {[}location{]} {[}下/上/里/\ldots{]}
  \item
    在桌子上 -\textgreater{} zài zhuōzi shàng -\textgreater{} on the
    table
  \item
    商店在我家旁边 -\textgreater{} shāngdiàn zài wǒ jiā pángbiān
    -\textgreater{} the shop is next to my house
  \end{itemize}
\item
  \textbf{在 + occasions}

  \begin{itemize}
  \tightlist
  \item
    structure: 在 {[}occasion{]} {[}上{]}
  \item
    我在派对上-\textgreater{} wǒ zài pàiduì shàng -\textgreater{} I'm at
    the party
  \item
    我在课上 -\textgreater{} I'm in class
  \end{itemize}
\end{itemize}

\hypertarget{relative-clauses-and-pronouns}{%
\subsection{relative clauses and
pronouns}\label{relative-clauses-and-pronouns}}

\begin{itemize}
\tightlist
\item
  if the relative clause has an object but no subject:

  \begin{itemize}
  \tightlist
  \item
    the object is the implied subject of the relative clause:
  \item
    学汉语的美国人 -\textgreater{} learn chinese {[}de{]} american
    -\textgreater{} the american \emph{who} learns chinese
  \end{itemize}
\item
  if the relative clause has a subject but no object:

  \begin{itemize}
  \tightlist
  \item
    the subject is the implied object of the relative clause:
  \item
    他说的英国 -\textgreater{} he speaks {[}de{]} chinese
    -\textgreater{} the chinese \emph{that} he speaks
  \end{itemize}
\item
  if the relative clause has both a subject and an object:

  \begin{itemize}
  \tightlist
  \item
    the direct object is the object of the relative clause
  \item
    我学汉语的美国人 -\textgreater{} I learn chinese {[}de{]} american
    -\textgreater{} I am an american who leans chinese
  \end{itemize}
\end{itemize}

\hypertarget{temporal}{%
\subsection{temporal}\label{temporal}}

\begin{itemize}
\item
  time words either go at the beginning of the sentence or directly
  after the subject

  \begin{itemize}
  \tightlist
  \item
    {[}time word{]} {[}S{]} {[}V{]} {[}O{]} -\textgreater{}
    昨天我去了酒吧 -\textgreater{} yesterday I went to the bar
  \item
    {[}S{]} {[}time word{]} {[}V{]} {[}O{]} -\textgreater{}
    我昨天去了酒吧 -\textgreater{} I went to the bar yesterday
  \end{itemize}
\item
  \textbf{以后 / yǐhòu -\textgreater{} after a specific time}

  \begin{itemize}
  \tightlist
  \item
    下午三点以后,我不在家 -\textgreater{} 3pm {[}after{]}, I will not
    be at home
  \item
    来中国以后,她认识了她的朋友 -\textgreater{} come china {[}after{]},
    she met her friend
  \end{itemize}
\item
  \textbf{以前 / yǐqián -\textgreater{} before a specific time}

  \begin{itemize}
  \tightlist
  \item
    吃饭以前,你洗手了马? -\textgreater{} eat {[}before{]}, you washed
    hand?
  \item
    睡觉以前,不要吃东西 -\textgreater{} sleep {[}before{]}, don't eat
    anything
  \end{itemize}
\item
  \textbf{现在 / xiànzài -\textgreater{} now}
\item
  \textbf{了 / le -\textgreater{} now}

  \begin{itemize}
  \tightlist
  \item
    in some common expressions 了 is used instead of 现在 to mean
    ``now''
  \item
    same pattern as the ``change in state'' usage of 了
  \item
    懂了 -\textgreater{} now I understand
  \item
    吃饭了! -\textgreater{} eat now! / time to eat!
  \item
    知道了 -\textgreater{} I understand now / got it / I see
  \item
    她来了 -\textgreater{} she's coming now
  \end{itemize}
\end{itemize}

\hypertarget{please-polite-requests}{%
\subsection{please / polite requests}\label{please-polite-requests}}

\begin{itemize}
\tightlist
\item
  \textbf{请 / qǐng} -\textgreater{} please

  \begin{itemize}
  \tightlist
  \item
    structure: {[}请{]} {[}V{]}
  \item
    请进 -\textgreater{} qǐng jìn -\textgreater{} please come in
  \item
    请说 -\textgreater{} qǐng shuō -\textgreater{} please speak
  \end{itemize}
\end{itemize}

\hypertarget{excessively-with-ux592a-and-ux4e86}{%
\subsection{excessively with 太 and
了}\label{excessively-with-ux592a-and-ux4e86}}

\begin{itemize}
\tightlist
\item
  can express something is too much with 太 and 了

  \begin{itemize}
  \tightlist
  \item
    structure: {[}太{]} {[}adj{]} {[}了{]}
  \item
    米饭太多了 -\textgreater{} mǐfàn tài duō le -\textgreater{} there is
    too much rice
  \item
    你太好了 -\textgreater{} nǐ tài hào le -\textgreater{} you are too
    good / you are so good
  \item
    seems you can use 太 in place of 很
  \end{itemize}
\end{itemize}

\hypertarget{ux90fd---all}{%
\subsection{都 -\textgreater{} all}\label{ux90fd---all}}

\begin{itemize}
\tightlist
\item
  你们都认识Shinzo马? -\textgreater{} you all know Shinzo?
\item
  明天我们都可以去 -\textgreater{} tomorrow we all can go
\item
  我们都要冰水 -\textgreater{} we all want ice water
\end{itemize}

\hypertarget{ux4e5f---alsotoo}{%
\subsection{也 -\textgreater{} also/too}\label{ux4e5f---alsotoo}}

\begin{itemize}
\item
  structure: {[}S{]} {[}也{]} {[}V{]}
\item
  我也喜欢 -\textgreater{} I also like it
\item
  她也有一个儿子 -\textgreater{} she also has a son
\item
  她也有一个女儿 -\textgreater{} she also has a daughter
\end{itemize}

\hypertarget{ux591a---how-much}{%
\subsection{多 -\textgreater{} how {[}much{]}}\label{ux591a---how-much}}

\begin{itemize}
\item
  多 -\textgreater{} how much. used for asking about the degree/extent
  of something
\item
  她多高?-\textgreater{} she {[}how much{]} tall? -\textgreater{} how
  tall is she?
\item
  你家多大?-\textgreater{} you house {[}how{]} big? -\textgreater{} how
  big is your house?
\end{itemize}

\hypertarget{determiners}{%
\subsection{determiners}\label{determiners}}

\begin{itemize}
\item
  \textbf{个 -\textgreater{} measure word}
\item
  \textbf{都 -\textgreater{} quantifier}
\item
  \textbf{什么 -\textgreater{} interrogative}
\item
  \textbf{examples:}
\item
  这是什么?

  \begin{itemize}
  \tightlist
  \item
    this is \emph{what?}
  \end{itemize}
\item
  他们是都不在家马?

  \begin{itemize}
  \tightlist
  \item
    they are \emph{all} not at home?
  \end{itemize}
\item
  那是什么手机?

  \begin{itemize}
  \tightlist
  \item
    that is what phone?
  \end{itemize}
\item
  你的是什么?

  \begin{itemize}
  \tightlist
  \item
    your{[}s{]} is what? (what is yours?)
  \end{itemize}
\item
  你的名字是什么?

  \begin{itemize}
  \tightlist
  \item
    your name is what?
  \end{itemize}
\item
  你叫什么?

  \begin{itemize}
  \tightlist
  \item
    you called what?
  \end{itemize}
\item
  你叫什么名字?

  \begin{itemize}
  \tightlist
  \item
    you called what name?
  \item
    {[}S{]}{[}V{]} {[}det{]} {[}O{]}
  \end{itemize}
\item
  你妈妈叫你什么名字?

  \begin{itemize}
  \tightlist
  \item
    your mother calls you what name?
  \item
    {[}S{]} {[}V{]} {[}IDO{]} {[}det{]} {[}DO{]}
  \end{itemize}
\end{itemize}

\hypertarget{imperatives}{%
\subsection{imperatives}\label{imperatives}}

\begin{itemize}
\tightlist
\item
  不要 / {[}bú yào{]} -\textgreater{} means ``don't'' {[}literally
  ``don't want''{]}

  \begin{itemize}
  \tightlist
  \item
    不要说英语 -\textgreater{} don't speak english
  \end{itemize}
\end{itemize}


\end{document}
