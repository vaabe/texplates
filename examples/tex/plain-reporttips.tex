%%%% PANDOC DEFAULTS %%%%

% Options for packages loaded elsewhere
\PassOptionsToPackage{unicode}{hyperref}
\PassOptionsToPackage{hyphens}{url}
%

\documentclass[
]{article}



\usepackage{lmodern}


\usepackage{amssymb,amsmath}
\usepackage{ifxetex,ifluatex}
\ifnum 0\ifxetex 1\fi\ifluatex 1\fi=0 % if pdftex
  \usepackage[T1]{fontenc}
  \usepackage[utf8]{inputenc}
  \usepackage{textcomp} % provide euro and other symbols
\else % if luatex or xetex

  \usepackage{unicode-math}

\defaultfontfeatures{Scale=MatchLowercase}
\defaultfontfeatures[\rmfamily]{Ligatures=TeX,Scale=1}









\fi


% Use upquote if available, for straight quotes in verbatim environments
\IfFileExists{upquote.sty}{\usepackage{upquote}}{}
\IfFileExists{microtype.sty}{% use microtype if available
  \usepackage[]{microtype}
  \UseMicrotypeSet[protrusion]{basicmath} % disable protrusion for tt fonts
}{}


\makeatletter
\@ifundefined{KOMAClassName}{% if non-KOMA class
  \IfFileExists{parskip.sty}{%
    \usepackage{parskip}
  }{% else
    \setlength{\parindent}{0pt}
    \setlength{\parskip}{6pt plus 2pt minus 1pt}}
}{% if KOMA class
  \KOMAoptions{parskip=half}}
\makeatother

\usepackage[dvipsnames]{xcolor}

\IfFileExists{xurl.sty}{\usepackage{xurl}}{} % add URL line breaks if available
\IfFileExists{bookmark.sty}{\usepackage{bookmark}}{\usepackage{hyperref}}

\urlstyle{same} % disable monospaced font for URLs











\setlength{\emergencystretch}{3em} % prevent overfull lines
\providecommand{\tightlist}{%
  \setlength{\itemsep}{0pt}\setlength{\parskip}{0pt}}
\setcounter{secnumdepth}{-\maxdimen} % remove section numbering










%%%% MAIN CONFIG %%%%

% title
\title{\textbf{Lab Report Tips}}

% subtitle

% author

% date
\date{}

% colors
\definecolor{MyBlue}{rgb}{0.0, 0.0, 0.75}

\hypersetup{
	colorlinks=true,
	linkcolor=MyBlue,
	urlcolor=MyBlue
}

% urlstyle
\urlstyle{tt}

% fonts
\usepackage{geometry}
\renewcommand{\familydefault}{\sfdefault}

\DeclareRobustCommand{\[}{\begin{equation}}
\DeclareRobustCommand{\]}{\end{equation}}

\begin{document}

% page margins
\newgeometry{
	margin = .2cm, 
	top = 2.0cm, 
	bottom = 2.0cm, 
	left = 2.0cm, 
	right = 2.0cm
}

\renewcommand{\v}[1]{{\mathbf{#1}}}
\newcommand{\dv}[1]{\dot{\mathbf{#1}}}
\newcommand{\ddv}[1]{\ddot{\mathbf{#1}}}
\newcommand{\hv}[1]{\hat{\mathbf{#1}}}
\newcommand{\m}[1]{[ #1 ]}

\newcommand{\bfit}[1]{\textbf{\textit{#1}}}
\renewcommand{\t}[1]{\text{#1}}

\renewcommand{\d}{\text{d}}
\newcommand{\dd}[2]{\frac{\d #1}{\d #2}}
\newcommand{\ddd}[2]{\frac{\d^2 #1}{\d #2^2}}
\newcommand{\ddt}[1]{\frac{\d #1}{\d t}}
\newcommand{\dddt}[1]{\frac{\d^2 #1}{\d t^2}}
\newcommand{\pd}[2]{\frac{\partial #1}{\partial #2}}
\newcommand{\pdd}[2]{\frac{\partial^2 #1}{\partial #2^2}}
\newcommand{\grad}{\mathbf \nabla} 
\renewcommand{\div}{\mathbf \nabla \cdot}
\newcommand{\curl}{\mathbf \nabla \times}
\newcommand{\lap}{\nabla^2}

\newcommand{\eo}{\epsilon_0}
\newcommand{\muo}{\mu_0}
\newcommand{\Lag}{\mathcal L}
\newcommand{\Ham}{\mathcal H}
\newcommand{\degc}{^\circ \text C}
\newcommand{\avo}{6.023 \cdot 10^{23}}
\renewcommand{\P}{\text{P}}
\newcommand{\p}{\text{p}}
\newcommand{\E}{\text{E}}
\newcommand{\e}[1]{\text{e}^{#1}}

\newcommand{\bra}[1]{\langle #1 |}
\newcommand{\ket}[1]{| #1 \rangle}
\newcommand{\braket}[2]{\langle #1 | #2 \rangle}
\newcommand{\adj}{^\dagger}
\newcommand{\cj}{^*}
\newcommand{\op}[1]{\hat{#1}}

\newcommand{\bm}{\begin{bmatrix}}
\newcommand{\ebm}{\end{bmatrix}}
\newcommand{\bal}{\begin{aligned}}
\newcommand{\eal}{\end{aligned}}
\newcommand{\eq}{\begin{equation}}
\newcommand{\eeq}{\end{equation}}
\newcommand{\tl}[2]{\tag{#1} \label{#1}}

\maketitle


%%%% TABLE OF CONTENTS %%%%


\hfill

\texttt{mdpdf\ plain-reporttips.md}

\hypertarget{stuff-i-generally-look-for}{%
\section{Stuff I Generally Look For}\label{stuff-i-generally-look-for}}

Here's the (rough) rubric I use when grading your report:

\begin{itemize}
\tightlist
\item
  \textbf{Theory: (2 points)}:

  \begin{itemize}
  \tightlist
  \item
    A writeup of the principles and equations used in the experiment.
  \item
    Your writeup should demonstrate that you understand the concepts.
    You should derive equations, show how they relate to each other,
    etc.
  \item
    Try to do more than just regurgitate from the manual. The manuals
    are an ok place to start but they're usually pretty light on detail.
    You should do your own reading/research. This semester I'll also try
    and post some resources for further reading.
  \end{itemize}
\item
  \textbf{Experiment setup: (2 points)}:

  \begin{itemize}
  \tightlist
  \item
    A description of the experiment.
  \item
    The goals, the variables you measured, the instruments used, the
    assumptions you made.
  \item
    Diagrams are good.
  \end{itemize}
\item
  \textbf{Presentation of data: (1-2 points)}:

  \begin{itemize}
  \tightlist
  \item
    Clear presentation of data.
  \item
    This includes tables, graphs, equations, etc (as applicable; not all
    experiments require all of these).
  \item
    Please label/signpost your content! Otherwise I won't know what I'm
    looking at. Perhaps contrary to popular belief, I am not an
    omnipotent God of The Labs. Although I've (probably) read the lab
    manual, by the time I get to grading your report I'll have basically
    forgotten everything about it. If you put a table in your report
    without labeling what it contains and what experiment it corresponds
    to, my initial reaction will probably be one of bafflement, not
    understanding.
  \end{itemize}
\item
  \textbf{Analysis, discussion: (2-3 points)}:

  \begin{itemize}
  \tightlist
  \item
    A discussion of your results. You should explain what they show, how
    they demonstrate the theory, etc.
  \item
    If applicable, compare your experimental values to the theoretical
    ones. Why are they different?
  \item
    You should answer the questions in the manual (marked in bold). It's
    up to you whether you integrate your answers into your general
    discussion, or if you answer them separately.
  \end{itemize}
\item
  \textbf{Sources of error: (2 points)}:

  \begin{itemize}
  \tightlist
  \item
    A discussion of factors that may have caused
    impresicions/inaccuracies in the results.
  \item
    Your sources of error should be relevant. Don't just randomly list
    ``ad hoc'' sources of error. You should explain why a particular
    source of error might account for the \emph{specific} discrepancies
    you see in the data.
  \item
    It's also worth nothing that sometimes the equipment we use in these
    labs is a bit old and terrible. If you can't figure out why some
    results are off, it may just be that the equipment was bad. Ask me
    if you're unsure.
  \item
    Feel free to critique the experiment design and suggest
    improvements.
  \end{itemize}
\item
  \textbf{Bibliography/references:}

  \begin{itemize}
  \tightlist
  \item
    I won't grade this, but if you use external sources (as indeed you
    should) you should state them.
  \end{itemize}
\end{itemize}

\hypertarget{dont-worry-too-much-about}{%
\section{Don't Worry Too Much
About\ldots{}}\label{dont-worry-too-much-about}}

\begin{itemize}
\item
  \textbf{Formatting:} beyond typing your report, you can use whatever
  style/format you want. Aesthetic frippery is not necessary. Simple and
  functional reports are completely fine.
\item
  \textbf{Grammar and writing style:} as long as I can basically follow
  what you're saying, we won't have trouble.
\end{itemize}

\hypertarget{error-analysis}{%
\section{Error Analysis}\label{error-analysis}}

Refer to
\href{https://physics.nyu.edu/~physlab/Lab_Main/Error\%20Analysis\%20for\%20premed\%20August\%2016,\%202010.pdf}{NYU's
guide on error analysis}.

\begin{itemize}
\item
  In general you should state the numerical uncertainties in
  experimental/measured values.
\item
  If you have an array of data (e.g.~if you have many measurements of a
  single variable or outcome) it's useful to give summary statistics
  (mean, standard deviation, etc).
\item
  Sometimes you will have to multiply experimental values together. When
  you do this, you may find it quite cumbersome to compute the
  uncertainty in the final value (the uncertainty becomes a quadratic
  function---see the error analysis guide). If you have a whole table of
  data it'll take you a very long time to do all these calculations by
  hand. I urge you to use software to do these calculations (for your
  own sake). If you don't know how, feel free to ask me. I will give you
  some code that helps speed things up for you. I want you to spend more
  time on analysis and building understanding than on repeating menial
  calculations!
\end{itemize}

\hypertarget{grading-errors}{%
\section{Grading Errors}\label{grading-errors}}

If you think I've graded something wrong or unfairly, please let me
know. I do miss things occasionally so I'm happy to resolve any issues
you have.


\end{document}
