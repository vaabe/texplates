%%%% PANDOC DEFAULTS %%%%

% Options for packages loaded elsewhere
\PassOptionsToPackage{unicode}{hyperref}
\PassOptionsToPackage{hyphens}{url}
%

\documentclass[
]{article}



\usepackage{lmodern}


\usepackage{amssymb,amsmath}
\usepackage{ifxetex,ifluatex}
\ifnum 0\ifxetex 1\fi\ifluatex 1\fi=0 % if pdftex
  \usepackage[T1]{fontenc}
  \usepackage[utf8]{inputenc}
  \usepackage{textcomp} % provide euro and other symbols
\else % if luatex or xetex

  \usepackage{unicode-math}

\defaultfontfeatures{Scale=MatchLowercase}
\defaultfontfeatures[\rmfamily]{Ligatures=TeX,Scale=1}









\fi


% Use upquote if available, for straight quotes in verbatim environments
\IfFileExists{upquote.sty}{\usepackage{upquote}}{}
\IfFileExists{microtype.sty}{% use microtype if available
  \usepackage[]{microtype}
  \UseMicrotypeSet[protrusion]{basicmath} % disable protrusion for tt fonts
}{}


\makeatletter
\@ifundefined{KOMAClassName}{% if non-KOMA class
  \IfFileExists{parskip.sty}{%
    \usepackage{parskip}
  }{% else
    \setlength{\parindent}{0pt}
    \setlength{\parskip}{6pt plus 2pt minus 1pt}}
}{% if KOMA class
  \KOMAoptions{parskip=half}}
\makeatother

\usepackage[dvipsnames]{xcolor}

\IfFileExists{xurl.sty}{\usepackage{xurl}}{} % add URL line breaks if available
\IfFileExists{bookmark.sty}{\usepackage{bookmark}}{\usepackage{hyperref}}

\urlstyle{same} % disable monospaced font for URLs









\usepackage{graphicx}
\usepackage{float}
\makeatletter
\def\maxwidth{\ifdim\Gin@nat@width>\linewidth\linewidth\else\Gin@nat@width\fi}
\def\maxheight{\ifdim\Gin@nat@height>\textheight\textheight\else\Gin@nat@height\fi}
\makeatother
% Scale images if necessary, so that they will not overflow the page
% margins by default, and it is still possible to overwrite the defaults
% using explicit options in \includegraphics[width, height, ...]{}
\setkeys{Gin}{width=\maxwidth,height=\maxheight,keepaspectratio}

% Set default figure placement to htbp
\makeatletter
\def\fps@figure{htbp}
\makeatother

\let\origfigure\figure
\let\endorigfigure\endfigure
\renewenvironment{figure}[1][2] {
    \expandafter\origfigure\expandafter[H]
} {
    \endorigfigure
}




\setlength{\emergencystretch}{3em} % prevent overfull lines
\providecommand{\tightlist}{%
  \setlength{\itemsep}{0pt}\setlength{\parskip}{0pt}}
\setcounter{secnumdepth}{5}










%%%% MAIN CONFIG %%%%

\title{\textbf{Resonance Tube}}

%
\author{VA}

\date{}

\definecolor{MyBlue}{rgb}{0.0, 0.0, 0.75}

\hypersetup{
	colorlinks=true,
	linkcolor=MyBlue,
	urlcolor=MyBlue
}

\urlstyle{tt}

\usepackage{geometry}
\renewcommand{\familydefault}{\sfdefault}

\DeclareRobustCommand{\[}{\begin{equation}}
\DeclareRobustCommand{\]}{\end{equation}}

\begin{document}

\newgeometry{
	margin = .2cm, 
	top = 1.8cm, 
	bottom = 1.8cm, 
	left = 1.6cm, 
	right = 1.6cm
}

\renewcommand{\v}[1]{{\mathbf{#1}}}
\newcommand{\dv}[1]{\dot{\mathbf{#1}}}
\newcommand{\ddv}[1]{\ddot{\mathbf{#1}}}
\newcommand{\hv}[1]{\hat{\mathbf{#1}}}
\newcommand{\m}[1]{[ #1 ]}

\newcommand{\bfit}[1]{\textbf{\textit{#1}}}
\renewcommand{\t}[1]{\text{#1}}

\renewcommand{\d}{\text{d}}
\newcommand{\dd}[2]{\frac{\d #1}{\d #2}}
\newcommand{\ddd}[2]{\frac{\d^2 #1}{\d #2^2}}
\newcommand{\ddt}[1]{\frac{\d #1}{\d t}}
\newcommand{\dddt}[1]{\frac{\d^2 #1}{\d t^2}}
\newcommand{\pd}[2]{\frac{\partial #1}{\partial #2}}
\newcommand{\pdd}[2]{\frac{\partial^2 #1}{\partial #2^2}}
\newcommand{\grad}{\mathbf \nabla} 
\renewcommand{\div}{\mathbf \nabla \cdot}
\newcommand{\curl}{\mathbf \nabla \times}
\newcommand{\lap}{\nabla^2}

\newcommand{\eo}{\epsilon_0}
\newcommand{\muo}{\mu_0}
\newcommand{\Lag}{\mathcal L}
\newcommand{\Ham}{\mathcal H}
\newcommand{\degc}{^\circ \text C}
\newcommand{\avo}{6.023 \cdot 10^{23}}
\renewcommand{\P}{\text{P}}
\newcommand{\p}{\text{p}}
\newcommand{\E}{\text{E}}
\newcommand{\e}[1]{\text{e}^{#1}}

\newcommand{\bra}[1]{\langle #1 |}
\newcommand{\ket}[1]{| #1 \rangle}
\newcommand{\braket}[2]{\langle #1 | #2 \rangle}
\newcommand{\adj}{^\dagger}
\newcommand{\cj}{^*}
\newcommand{\op}[1]{\hat{#1}}

\newcommand{\bm}{\begin{bmatrix}}
\newcommand{\ebm}{\end{bmatrix}}
\newcommand{\bal}{\begin{aligned}}
\newcommand{\eal}{\end{aligned}}
\newcommand{\eq}{\begin{equation}}
\newcommand{\eeq}{\end{equation}}
\newcommand{\tl}[2]{\tag{#1} \label{#1}}

\maketitle


%%%% TABLE OF CONTENTS %%%%


{
\setcounter{tocdepth}{3}
\tableofcontents
}

\hfill

\texttt{mdpdf\ labmanual-resonancetube.md\ -t\ la}

\hypertarget{theory}{%
\section{Theory}\label{theory}}

\hypertarget{standing-waves-in-tubes-basics}{%
\subsection{Standing waves in tubes:
basics}\label{standing-waves-in-tubes-basics}}

Here's a (very) brief summary of the theory:

When sound waves travel through a finite tube, the waves get reflected
at the ends of the tube. The incident waves interfere with the reflected
waves, and the resulting wave pattern can be found by adding the
displacements of the individual waves (principle of superposition).

At certain frequencies, the incident waves and reflected waves interfere
constructively, producing a wave pattern with distinct \textbf{nodes}
(points that don't move) and \textbf{antinodes} (points of maximum
oscillation). These are called \textbf{standing waves} (so called
because the wave doesn't appear to be moving/propagating in any
particular direction---it just oscillates). The frequencies that produce
this standing wave pattern are called \textbf{resonant frequencies.}

The tube's boundary conditions (i.e.~whether the tube is open or closed
at the ends) will determine the kind of standing wave that can be
produced. If the tube has a closed end, that end must necessarily be a
node (since particles are fixed in place). If a tube has an open end,
that end must be an antinode (since particles are free to oscillate).

\begin{figure}
\centering
\includegraphics[width=0.8\textwidth,height=\textheight]{./figs/modes-diagram.jpg}
\caption{Resonance states for open and closed tubes}
\end{figure}

The length of the tube also affects the the kind of standing waves that
can be produced. A standing wave can only occur if a periodic
``element'' of the wave fits perfectly inside the length of the tube. In
a fully closed tube, there must be a node at each end in order for a
standing wave to be observed. In a tube with one end open, there must be
a node at the closed end and an antinode at the other. In a fully open
tube, there must be antinodes at both ends.

The simplest frequency that satisfies these requirements is called the
fundamental harmonic. Each successive ``higher-order'' harmonic will
have one extra set of nodes and antinodes over the previous one.

\hypertarget{further-reading-references-videos}{%
\subsection{Further reading, references,
videos}\label{further-reading-references-videos}}

Some videos:

\begin{itemize}
\tightlist
\item
  \href{https://www.youtube.com/watch?v=gT0IqL1dyyk}{Khan Academy,
  ``Standing Waves''}
\item
  \href{https://www.youtube.com/watch?v=BhQUW9s-R8M}{Khan Academy,
  ``Standing Waves in Open Tubes''}
\item
  \href{https://www.youtube.com/watch?v=1S4DtuMY88I}{Khan Academy,
  ``Standing Waves in Closed Tubes''}
\end{itemize}

Some reading material:

\begin{itemize}
\tightlist
\item
  \href{https://openstax.org/books/university-physics-volume-1/pages/16-6-standing-waves-and-resonance}{Openstax
  Physics, Standing Waves and Resonance}
\item
  \href{https://drive.google.com/drive/folders/1s5rGFl1JyXJO0UvyvqE3euCgO9LsVNT0}{Young
  \& Freedman. University Physics, Ch 15, ``Mechanical Waves''.}
\item
  \href{https://www.feynmanlectures.caltech.edu/I_47.html}{Feynman
  Lectures, Ch47, ``Sound. The wave equation''}
\item
  \href{https://www.feynmanlectures.caltech.edu/I_49.html}{Feynman
  Lectures, Ch 49, ``Modes''}
\end{itemize}

\hypertarget{experiment-setup}{%
\section{Experiment Setup}\label{experiment-setup}}

\hypertarget{overview}{%
\subsection{Overview}\label{overview}}

In this lab you will examine standing sound waves in a tube with
different boundary conditions and lengths. One end of the tube is always
closed, and has a speaker attached to it. The other end of the tube can
be closed with a piston which you can move back and forth to change the
length of the tube. You can also remove the piston to create a tube with
one end open. Below is a horizontal view of the apparatus:

\begin{figure}
\centering
\includegraphics[width=1\textwidth,height=\textheight]{./figs/tube-fig-1.png}
\caption{Horizontal view of resonance tube}
\end{figure}

And below is a diagram of the individual parts:

\begin{figure}
\centering
\includegraphics[width=1\textwidth,height=\textheight]{./figs/tube-fig-2.png}
\caption{Apparatus parts}
\end{figure}

In the first experiment, the speaker is used to to send sinusoidal sound
waves through the tube. A small microphone in the tube detects the sound
waves. At the resonant frequencies, the sound waves will be enhanced.

In the second experiment, a pulse of sound is sent down the tube and the
reflected pulse will be examined. The nature of the reflected pulse
depends on whether the end of the tube is open or closed. You can use
the observed and reflected pulses to estimate the speed of sound.

\hypertarget{hardware-setup}{%
\subsection{Hardware setup}\label{hardware-setup}}

\begin{enumerate}
\def\labelenumi{\arabic{enumi}.}
\item
  At the speaker end of the tube, notice that there is a small
  microphone-shaped circular hole. \textbf{Move the microphone into the
  tube and place it at the 10 cm mark.} You may need to loosen the
  thumbscrew to allow the microphone into the tube.
\item
  Notice that the lead from the microphone goes to a preamplifier which
  has an on-off switch. \textbf{Turn this switch on.}
\item
  Make sure the rest of the setup is wired correctly. You should find
  that the output of the preamplifier connects to a phone plug, which
  then connects to a coaxial adapter, which then connects to a BNC
  adapter, and which reaches its logical conclusion at the banana plugs
  of a voltage sensor:
\item
  Connect the voltage sensor to Channel A of the Pasco 850 interface.
\item
  Connect the two leads of the speaker are connected to Output 1 of the
  '850.
\end{enumerate}

\hypertarget{capstone-setup}{%
\subsection{Capstone setup}\label{capstone-setup}}

\begin{enumerate}
\def\labelenumi{\arabic{enumi}.}
\item
  Open Capstone.
\item
  In the \emph{Hardware Setup} pane, click on Channel A and select
  ``Voltage Sensor''.
\item
  Click on the gear icon in the bottom right to open the voltage sensor
  properties.
\end{enumerate}

\includegraphics[width=0.6\textwidth,height=\textheight]{./figs/voltage-sensor-properties.png}

\begin{enumerate}
\def\labelenumi{\arabic{enumi}.}
\setcounter{enumi}{3}
\tightlist
\item
  In the voltage sensor properties window, change the sensitivity (gain)
  from the default of 1x to 100x.
\end{enumerate}

\includegraphics[width=0.55\textwidth,height=\textheight]{./figs/change-gain.png}

\begin{enumerate}
\def\labelenumi{\arabic{enumi}.}
\setcounter{enumi}{4}
\item
  Back in the Hardware Setup pane, click on Output 1 of the '850 and
  select ``Output Voltage - Current Sensor''
\item
  In the \emph{Displays} pane (on the RHS of the display), click and
  drag the ``Scope'' icon into the center of the screen.
\item
  For the \(y\)-axis, under ``Select Measurements'', choose ``Output
  Voltage Ch 01''.
\item
  At the top of the scope window, click the icon for ``Add new
  \(y\)-axis'', to add the second channel. For this second \(y\)-axis
  (which should appear on the RHS), choose ``Voltage, Ch A''.
\end{enumerate}

\includegraphics[width=1\textwidth,height=\textheight]{./figs/setting-up-scope.png}

The Capstone oscillscope should now be configured. When you hit
``Monitor'' you should see two traces appear on the scope. The voltage
of the signal generator (the signal being emitted from the speaker)
should be on the first channel, and the voltage from the voltage sensor
(the signal that's being detected by the mic) should be on the second
channel.

\hypertarget{features-of-the-capstone-scope}{%
\subsection{Features of the Capstone
scope}\label{features-of-the-capstone-scope}}

The Capstone oscilloscope is designed to display the waveform of the
sound waves. The \(x\)-axis is time, and the \(y\)-axis is voltage
output.

The image below indicates some (useful) features of the Capstone
oscilloscope:

\includegraphics[width=1\textwidth,height=\textheight]{./figs/the-capstone-scope.png}

\begin{itemize}
\tightlist
\item
  \textbf{A.} click and drag the \(y\)-axis in the vertical direction to
  change the scale
\item
  \textbf{B.} moves the scope trace up or down on the screen
\item
  \textbf{C.} click and drag the \(x\)-axis in the horizontal direction
  to stretch or compress the trace
\item
  \textbf{D.} selects the ``smart tool'' function
\item
  \textbf{E, F.} the trigger. This is used to make the scope trace
  appear static on the screen. In particular, if you notice the trace is
  ``moving'' across the screen, hit the trigger button. Move the arrow
  (marked \textbf{F}) up or down to freeze the trace, and/or adjust the
  trigger level (the value at which the sweep will start the signal).
\item
  \textbf{G.} adjust the \(y\)-axis to fit the data. This is useful to
  see the full amplitude of multiple waveforms.
\end{itemize}

\hypertarget{experiment-1-measuring-wavelength}{%
\section{Experiment 1: Measuring
Wavelength}\label{experiment-1-measuring-wavelength}}

\hypertarget{overview-1}{%
\subsection{Overview}\label{overview-1}}

In this experiment you will attempt to measure the wavelength of the
standing sound wave in the tube. You will do this by changing the
position of the piston (i.e.~altering the length of the tube) and noting
the positions that correspond to resonant frequencies.

\hypertarget{setup-procedures}{%
\subsection{Setup procedures}\label{setup-procedures}}

\begin{enumerate}
\def\labelenumi{\arabic{enumi}.}
\tightlist
\item
  Go to the \emph{Signal Generator} pane (on the LHS) and click on ``850
  Output 1''. Under the ``Waveform'' dropdown, select ``sine wave''. Set
  the frequency to 650 Hz and the amplitude to 1.1 V. And click the
  button marked ``Auto''.
\end{enumerate}

\includegraphics[width=0.5\textwidth,height=\textheight]{./figs/configure-signal-generator.png}

\begin{enumerate}
\def\labelenumi{\arabic{enumi}.}
\setcounter{enumi}{1}
\tightlist
\item
  At the bottom of the screen, select ``Fast Monitor Mode''.
\end{enumerate}

\includegraphics[width=0.25\textwidth,height=\textheight]{./figs/fast-monitor-mode.png}

\begin{enumerate}
\def\labelenumi{\arabic{enumi}.}
\setcounter{enumi}{2}
\tightlist
\item
  Adjust the trigger arrow and move it slightly above 0.2 V. Then
  ``click on adjust the y axis scale''.
\end{enumerate}

\hypertarget{experiment-procedures}{%
\subsection{Experiment procedures}\label{experiment-procedures}}

\begin{enumerate}
\def\labelenumi{\arabic{enumi}.}
\item
  First, move the piston into the tube, until it's positioned near the
  mic.
\item
  Then gradually pull the piston outwards, (i.e.~increasing the
  effective length of the tube).
\item
  As you do, watch how the waveform changes on Capstone. Find the
  positions of the piston that give the maximum signal from the
  microphone. Make note of these positions. These are the positions that
  correspond to resonant frequencies.
\item
  Note that successive piston positions are half a wavelength apart.
  Determine the wavelength \(\lambda\) using your data.
\item
  Compare the experimental value of \(\lambda\) to the theoretical one,
  which is given by \(\lambda f = v\).
\item
  Repeat the above procedures for signal frequencies of 900 Hz and 1300
  Hz. Do the same analysis. How does the wavelength change when you
  increase or decrease the frequency? Explain your results.
\end{enumerate}

\hypertarget{experiment-2-pulsed-waves}{%
\section{Experiment 2: Pulsed Waves}\label{experiment-2-pulsed-waves}}

\hypertarget{overview-2}{%
\subsection{Overview}\label{overview-2}}

In this experiment, short pulses will be sent down the tube from the
speaker. This is achieved by driving the speaker with a relatively low
frequency of 8 Hz. The will cause the speaker cone to move very quickly
one way and then stop, which sends a short compression or rarefraction
pulse down the tube. This pulse will reflect from the end of the tube
and propagates back towards the speaker. It will then bounce off the
speaker end of the tube, and continue to go back and forth along the
tube until it's damped. The period of the square wave is chosen to be
long enough so that the pulse is completely dissipated before the
speaker cone moves the other way in response to the square wave, and
sends the next pulse down the tube. The pulses alternate between
compressions and rarefractions.

\hypertarget{procedures}{%
\subsection{Procedures}\label{procedures}}

\begin{enumerate}
\def\labelenumi{\arabic{enumi}.}
\item
  Configure the signal generator for an 8 Hz positive square wave with
  an amplitude of 3 V.
\item
  Loosen the thumbscrew holding the microphone, and place the microphone
  in the tube at the 10 cm mark.
\item
  Move the piston out to 80 cm. Use the speaker stand to support the
  tube.
\item
  Make sure Capstone is set to ``Fast monitor mode''.
\item
  Next, you'll want to adjust the scope to obtain the initial pulse and
  the reflected pulse. There are two possible ways you can do this:

  \textbf{Method 1:}

  \begin{enumerate}
  \def\labelenumii{\arabic{enumii}.}
  \item
    Make sure you've clicked on the trigger button and adjusted the
    trigger arrow to be slightly positive (just as in experiment 1).
  \item
    Click ``Monitor''.
  \item
    Click on ``adjust \(y\)-axis scale to fit data'' (top left). You
    should see an image similar the one below. The region highlighted in
    green is the initial pulse and the region highlighted in blue is the
    reflected pulse.

    \includegraphics[width=0.8\textwidth,height=\textheight]{./figs/method-1.png}
  \end{enumerate}

  \textbf{Method 2:}

  \begin{enumerate}
  \def\labelenumii{\arabic{enumii}.}
  \item
    Adjust the \(x\)-axis (by dragging it horizontally) so the scope
    pattern looks like the one in the image below. There should be about
    0.05 s/div.

    \includegraphics[width=0.8\textwidth,height=\textheight]{./figs/method-2.png}
  \item
    On the scope click on the \(x\)-axis variable (time) and adjust the
    time division to approximately 1 ms/div, so that the trace looks
    like the image below:

    \includegraphics[width=0.8\textwidth,height=\textheight]{./figs/method-2a.png}
  \end{enumerate}
\end{enumerate}

\hypertarget{determining-the-speed-of-sound}{%
\subsection{Determining the speed of
sound}\label{determining-the-speed-of-sound}}

By using the displacement between the microphone and piston, and
determining the time between the initial pulse and the reflected pulse,
you can determine the speed of sound. Use the \emph{Delta Tool}
(Capstone) to find the time between pulses:

\includegraphics[width=0.4\textwidth,height=\textheight]{./figs/delta-tool.png}

You may want to increase the number of decimal places to obtain a
precise value for the time between pulses. In the \emph{Tools} column,
click on ``Data Summary'', go to ``clock'', and click on the gear icon.
A properties window will show up:

\includegraphics[width=0.6\textwidth,height=\textheight]{./figs/data-summary.png}

With the scope running, slowly move the piston towards the speaker. What
is occurring to the signal on the scope? Explain.

\hypertarget{changing-the-boundary-conditions}{%
\subsection{Changing the boundary
conditions}\label{changing-the-boundary-conditions}}

In a resonance tube, the nature of the reflected pulse depends on the
boundary conditions. If the end of the tube is open, it becomes a
pressure node, which causes the pressure of the reflected pulse to
invert, and cancel out the pressure of the initial pulse.

If the end of the tube is closed (as in experiment 1), the boundary is a
displacement node and a pressure antinode, so the pressure of the
reflected pulse is not inverted.

Remove the piston from the tube and support the end of the tube with the
stand. The tube should now have one end open. Hit ``Monitor'' and
observe the reflected pulse.

Now close the end of the tube with the rubber stopper and note how the
reflected pulse changes. Discuss what you see.


\end{document}
