%%%% PANDOC DEFAULTS %%%%

% Options for packages loaded elsewhere
\PassOptionsToPackage{unicode}{hyperref}
\PassOptionsToPackage{hyphens}{url}
%

\documentclass[
]{article}



\usepackage{lmodern}


\usepackage{amssymb,amsmath}
\usepackage{ifxetex,ifluatex}
\ifnum 0\ifxetex 1\fi\ifluatex 1\fi=0 % if pdftex
  \usepackage[T1]{fontenc}
  \usepackage[utf8]{inputenc}
  \usepackage{textcomp} % provide euro and other symbols
\else % if luatex or xetex

  \usepackage{unicode-math}

\defaultfontfeatures{Scale=MatchLowercase}
\defaultfontfeatures[\rmfamily]{Ligatures=TeX,Scale=1}









\fi


% Use upquote if available, for straight quotes in verbatim environments
\IfFileExists{upquote.sty}{\usepackage{upquote}}{}
\IfFileExists{microtype.sty}{% use microtype if available
  \usepackage[]{microtype}
  \UseMicrotypeSet[protrusion]{basicmath} % disable protrusion for tt fonts
}{}


\makeatletter
\@ifundefined{KOMAClassName}{% if non-KOMA class
  \IfFileExists{parskip.sty}{%
    \usepackage{parskip}
  }{% else
    \setlength{\parindent}{0pt}
    \setlength{\parskip}{6pt plus 2pt minus 1pt}}
}{% if KOMA class
  \KOMAoptions{parskip=half}}
\makeatother

\usepackage[dvipsnames]{xcolor}

\IfFileExists{xurl.sty}{\usepackage{xurl}}{} % add URL line breaks if available
\IfFileExists{bookmark.sty}{\usepackage{bookmark}}{\usepackage{hyperref}}

\urlstyle{same} % disable monospaced font for URLs











\setlength{\emergencystretch}{3em} % prevent overfull lines
\providecommand{\tightlist}{%
  \setlength{\itemsep}{0pt}\setlength{\parskip}{0pt}}
\setcounter{secnumdepth}{-\maxdimen} % remove section numbering










%%%% MAIN CONFIG %%%%

% title

% subtitle

% author

% date
\date{}

% colors
\definecolor{MyBlue}{rgb}{0.0, 0.0, 0.75}

\hypersetup{
	colorlinks=true,
	linkcolor=MyBlue,
	urlcolor=MyBlue
}

% urlstyle
\urlstyle{tt}

% fonts
\usepackage{geometry}
\renewcommand{\familydefault}{\sfdefault}

\DeclareRobustCommand{\[}{\begin{equation}}
\DeclareRobustCommand{\]}{\end{equation}}

\begin{document}

% page margins
\newgeometry{
	margin = .2cm, 
	top = 2.0cm, 
	bottom = 2.0cm, 
	left = 2.0cm, 
	right = 2.0cm
}

\renewcommand{\v}[1]{{\mathbf{#1}}}
\newcommand{\dv}[1]{\dot{\mathbf{#1}}}
\newcommand{\ddv}[1]{\ddot{\mathbf{#1}}}
\newcommand{\hv}[1]{\hat{\mathbf{#1}}}
\newcommand{\m}[1]{[ #1 ]}

\newcommand{\bfit}[1]{\textbf{\textit{#1}}}
\renewcommand{\t}[1]{\text{#1}}

\renewcommand{\d}{\text{d}}
\newcommand{\dd}[2]{\frac{\d #1}{\d #2}}
\newcommand{\ddd}[2]{\frac{\d^2 #1}{\d #2^2}}
\newcommand{\ddt}[1]{\frac{\d #1}{\d t}}
\newcommand{\dddt}[1]{\frac{\d^2 #1}{\d t^2}}
\newcommand{\pd}[2]{\frac{\partial #1}{\partial #2}}
\newcommand{\pdd}[2]{\frac{\partial^2 #1}{\partial #2^2}}
\newcommand{\grad}{\mathbf \nabla} 
\renewcommand{\div}{\mathbf \nabla \cdot}
\newcommand{\curl}{\mathbf \nabla \times}
\newcommand{\lap}{\nabla^2}

\newcommand{\eo}{\epsilon_0}
\newcommand{\muo}{\mu_0}
\newcommand{\Lag}{\mathcal L}
\newcommand{\Ham}{\mathcal H}
\newcommand{\degc}{^\circ \text C}
\newcommand{\avo}{6.023 \cdot 10^{23}}
\renewcommand{\P}{\text{P}}
\newcommand{\p}{\text{p}}
\newcommand{\E}{\text{E}}
\newcommand{\e}[1]{\text{e}^{#1}}

\newcommand{\bra}[1]{\langle #1 |}
\newcommand{\ket}[1]{| #1 \rangle}
\newcommand{\braket}[2]{\langle #1 | #2 \rangle}
\newcommand{\adj}{^\dagger}
\newcommand{\cj}{^*}
\newcommand{\op}[1]{\hat{#1}}

\newcommand{\bm}{\begin{bmatrix}}
\newcommand{\ebm}{\end{bmatrix}}
\newcommand{\bal}{\begin{aligned}}
\newcommand{\eal}{\end{aligned}}
\newcommand{\eq}{\begin{equation}}
\newcommand{\eeq}{\end{equation}}
\newcommand{\tl}[2]{\tag{#1} \label{#1}}



%%%% TABLE OF CONTENTS %%%%


\hfill

\hypertarget{sonometer-lab-instructions}{%
\section{Sonometer Lab Instructions}\label{sonometer-lab-instructions}}

\texttt{mdpdf\ plain-sonometer.md}

The point of the lab is to demonstrate how the normal mode frequency of
a string depends on length, tension, and mass density. Read through the
manual. You'll see there are four experiments in this lab:

\begin{enumerate}
\def\labelenumi{\arabic{enumi}.}
\tightlist
\item
  Mode frequency versus \(n\)
\item
  First mode frequency versus length \(L\)
\item
  First mode frequency versus tension \(T\)
\item
  Mode frequencies and mass density \(\mu\)
\end{enumerate}

The experiments have been done for you by a member of the physics
faculty. Your task is to write a lab report based on their data, which
we've provided for you in the data folder. I've also posted an
interactive version of the data on Airtable, which you may find is
easier to work with---follow the link:

\url{https://airtable.com/shrhqwPjVGr3WzBp7}

If you're stuck on how to write the report, read over the report writing
guidelines I've posted. It's not a rulebook, but it outlines some of the
basic components your report should probably have.

Report 1 is due on Thursday February 25th (not the 18th as stated
earlier). If you have questions please let me know.


\end{document}
