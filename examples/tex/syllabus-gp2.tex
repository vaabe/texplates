%%%% PANDOC DEFAULTS %%%%

% Options for packages loaded elsewhere
\PassOptionsToPackage{unicode}{hyperref}
\PassOptionsToPackage{hyphens}{url}
%

\documentclass[
]{article}



\usepackage{lmodern}


\usepackage{amssymb,amsmath}
\usepackage{ifxetex,ifluatex}
\ifnum 0\ifxetex 1\fi\ifluatex 1\fi=0 % if pdftex
  \usepackage[T1]{fontenc}
  \usepackage[utf8]{inputenc}
  \usepackage{textcomp} % provide euro and other symbols
\else % if luatex or xetex

  \usepackage{unicode-math}

\defaultfontfeatures{Scale=MatchLowercase}
\defaultfontfeatures[\rmfamily]{Ligatures=TeX,Scale=1}









\fi


% Use upquote if available, for straight quotes in verbatim environments
\IfFileExists{upquote.sty}{\usepackage{upquote}}{}
\IfFileExists{microtype.sty}{% use microtype if available
  \usepackage[]{microtype}
  \UseMicrotypeSet[protrusion]{basicmath} % disable protrusion for tt fonts
}{}


\makeatletter
\@ifundefined{KOMAClassName}{% if non-KOMA class
  \IfFileExists{parskip.sty}{%
    \usepackage{parskip}
  }{% else
    \setlength{\parindent}{0pt}
    \setlength{\parskip}{6pt plus 2pt minus 1pt}}
}{% if KOMA class
  \KOMAoptions{parskip=half}}
\makeatother

\usepackage[dvipsnames]{xcolor}

\IfFileExists{xurl.sty}{\usepackage{xurl}}{} % add URL line breaks if available
\IfFileExists{bookmark.sty}{\usepackage{bookmark}}{\usepackage{hyperref}}

\urlstyle{same} % disable monospaced font for URLs







\usepackage{longtable,booktabs}
% Correct order of tables after \paragraph or \subparagraph
\usepackage{etoolbox}
\makeatletter
\patchcmd\longtable{\par}{\if@noskipsec\mbox{}\fi\par}{}{}
\makeatother
% Allow footnotes in longtable head/foot
\IfFileExists{footnotehyper.sty}{\usepackage{footnotehyper}}{\usepackage{footnote}}
\makesavenoteenv{longtable}




\setlength{\emergencystretch}{3em} % prevent overfull lines
\providecommand{\tightlist}{%
  \setlength{\itemsep}{0pt}\setlength{\parskip}{0pt}}
\setcounter{secnumdepth}{-\maxdimen} % remove section numbering










%%%% MAIN CONFIG %%%%

% title
\title{\textbf{Syllabus: GP2 Lab}}

% subtitle
\usepackage{etoolbox}
\makeatletter
\providecommand{\subtitle}[1]{% add subtitle to \maketitle
  \apptocmd{\@title}{\par {\large #1 \par}}{}{}
}
\makeatother
\subtitle{Spring 2021}

%% semester
%%\subtitle{}

%% author
%
% date
\date{}

% colors
\definecolor{MyBlue}{rgb}{0.0, 0.0, 0.75}

\hypersetup{
	colorlinks=true,
	linkcolor=MyBlue,
	urlcolor=MyBlue
}

% urlstyle
\urlstyle{tt}

\usepackage{geometry}
\renewcommand{\familydefault}{\sfdefault}

\DeclareRobustCommand{\[}{\begin{equation}}
\DeclareRobustCommand{\]}{\end{equation}}

\begin{document}

% page margins
\newgeometry{
	margin=.2cm, 
	top=2.0cm, 
	bottom=2.0cm, 
	left=1.6cm, 
	right=1.6cm
}

% newcommands
\renewcommand{\v}[1]{{\mathbf{#1}}}
\newcommand{\dv}[1]{\dot{\mathbf{#1}}}
\newcommand{\ddv}[1]{\ddot{\mathbf{#1}}}
\newcommand{\hv}[1]{\hat{\mathbf{#1}}}
\newcommand{\m}[1]{[ #1 ]}

\newcommand{\bfit}[1]{\textbf{\textit{#1}}}
\renewcommand{\t}[1]{\text{#1}}

\renewcommand{\d}{\text{d}}
\newcommand{\dd}[2]{\frac{\d #1}{\d #2}}
\newcommand{\ddd}[2]{\frac{\d^2 #1}{\d #2^2}}
\newcommand{\ddt}[1]{\frac{\d #1}{\d t}}
\newcommand{\dddt}[1]{\frac{\d^2 #1}{\d t^2}}
\newcommand{\pd}[2]{\frac{\partial #1}{\partial #2}}
\newcommand{\pdd}[2]{\frac{\partial^2 #1}{\partial #2^2}}
\newcommand{\grad}{\mathbf \nabla} 
\renewcommand{\div}{\mathbf \nabla \cdot}
\newcommand{\curl}{\mathbf \nabla \times}
\newcommand{\lap}{\nabla^2}

\newcommand{\eo}{\epsilon_0}
\newcommand{\muo}{\mu_0}
\newcommand{\Lag}{\mathcal L}
\newcommand{\Ham}{\mathcal H}
\newcommand{\degc}{^\circ \text C}
\newcommand{\avo}{6.023 \cdot 10^{23}}
\renewcommand{\P}{\text{P}}
\newcommand{\p}{\text{p}}
\newcommand{\E}{\text{E}}
\newcommand{\e}[1]{\text{e}^{#1}}

\newcommand{\bra}[1]{\langle #1 |}
\newcommand{\ket}[1]{| #1 \rangle}
\newcommand{\braket}[2]{\langle #1 | #2 \rangle}
\newcommand{\adj}{^\dagger}
\newcommand{\cj}{^*}
\newcommand{\op}[1]{\hat{#1}}

\newcommand{\bm}{\begin{bmatrix}}
\newcommand{\ebm}{\end{bmatrix}}
\newcommand{\bal}{\begin{aligned}}
\newcommand{\eal}{\end{aligned}}
\newcommand{\eq}{\begin{equation}}
\newcommand{\eeq}{\end{equation}}
\newcommand{\tl}[2]{\tag{#1} \label{#1}}

\maketitle
%
%% table of contents
%%
%\hfill

\texttt{mdpdf\ syllabus-gp2.md\ -t\ sy}

\hypertarget{info}{%
\section{Info}\label{info}}

\begin{itemize}
\item
  \textbf{Contact:} \href{mailto:vabe@nyu.edu}{\nolinkurl{vabe@nyu.edu}}
\item
  \textbf{Room:} Meyer 222
\item
  \textbf{Class hours:}

  \begin{itemize}
  \tightlist
  \item
    Sec 07: Mon 1.30pm-3.50pm
  \item
    Sec 11: Mon 4.00pm-6.20pm
  \item
    Sec 04: Thu 11.00am-1.20pm
  \item
    Sec 08: Thu 1.30pm-3.50pm
  \end{itemize}
\item
  \textbf{Office hours:} \url{https://nyu.zoom.us/j/91270104640}

  \begin{itemize}
  \tightlist
  \item
    Tue/Wed/Fri 4.00pm-5.00pm ET (or by appointment---please ask)
  \end{itemize}
\item
  \textbf{Github:} \url{https://github.com/vaabe/phys12}
\end{itemize}

\hypertarget{description}{%
\section{Description}\label{description}}

This laboratory course is intended to help you understand the basic
principles of waves, electromagnetism, and optics. There are 10 labs in
total:

\begin{longtable}[]{@{}ll@{}}
\toprule
\endhead
Feb 11 & 1 - Sonometer \\
Feb 18 & \\
Feb 25 & 2 - Resonance Tube \\
Mar 4 & 3 - Electrostatics \\
Mar 11 & 4 - Electric Field Mapping \\
Mar 18 & \\
Mar 25 & 5 - Voltage, Current, Resistance \\
Apr 1 & 6 - Charge to Mass Ratio of Electron \\
Apr 8 & 7 - Current Balance \\
Apr 15 & 8 - Induction \\
Apr 22 & \\
Apr 29 & 9 - Snell's Law \\
May 4 & 10 - Eye \\
\bottomrule
\end{longtable}

\hypertarget{in-person-vs-remote}{%
\section{In-Person vs Remote}\label{in-person-vs-remote}}

You choose at the beginning of the semester whether you're going to take
this course in-person or remotely. The department's policy is that you
should stick to your choice throughout the semester (to the extent that
you can; obviously we'll make reasonable exceptions due to covid).

\hypertarget{in-person-labs}{%
\section{In-Person Labs}\label{in-person-labs}}

If you're taking the course in-person, you will perform the experiments
individually. You should read the manual before coming to class so you
know what to expect. {[}At the very least you should skim over the
theory. E\&M is non-trivial and it'll be easier if you actually
understand what you're observing{]}.

\hypertarget{remote-labs}{%
\section{Remote Labs}\label{remote-labs}}

Each week we'll post the relevant lab materials (data, videos,
simulations, etc) in the
\href{https://drive.google.com/drive/folders/1onTfWs8QGWsOP_3PFigj7CvMpJUr_Kn4?usp=sharing}{google
drive folder}. You should base your report on the data we provide. There
will be two or three zoom office hours each week. These meetings aren't
mandatory but I encourage you to show up and ask questions. Asides from
that you'll be working fairly independently, and it's your
responsibility to manage your time and keep up with the schedule.

\hypertarget{assignments}{%
\section{Assignments}\label{assignments}}

Each week you will write a report on the experiment you performed that
week. I will mark your submission out of 10:

\begin{itemize}
\tightlist
\item
  \textbf{Theory: /2}
\item
  \textbf{Experiment Setup: /2}
\item
  \textbf{Results: /6} (split into three subsections)

  \begin{itemize}
  \tightlist
  \item
    \textbf{Presentation of data: /2}
  \item
    \textbf{Analysis of results, discussion questions: /2}
  \item
    \textbf{Sources of error: /2}
  \end{itemize}
\end{itemize}

I've also posted a guide with more detailed advice on writing reports.
Give it a read if you're stumped for ideas.

\hypertarget{deadlines}{%
\section{Deadlines}\label{deadlines}}

Reports are due at the beginning of the next scheduled session
(i.e.~Thursdays). Please be reasonable.

\hypertarget{resubmission}{%
\section{Resubmission}\label{resubmission}}

Throughout the semester I'll allow you to resubmit two reports for a
better grade. Email me.


\end{document}
